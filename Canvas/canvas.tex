\documentclass[10pt]{beamer}

\input{/Users/daniel/Documents/LaTeX/beamer-style.tex}

\title{Multimédia - 2\textsuperscript{e}}
\subtitle{Introduction à Canvas}
\date{\today}
\author{Daniel Schreurs}
\institute{Haute École de Province de Liège}
%\titlegraphic{\hfill\includegraphics[height=1.5cm]{logo.eps}}


\setbeamertemplate{frame footer}{\insertsectionhead}
\begin{document}
\maketitle

\setbeamerfont{subsection in toc}{size=\small}
\setbeamerfont{subsubsection in toc}{size=\normalsize}
\setbeamertemplate{section in toc}[sections numbered]
\setbeamertemplate{subsection in toc}[subsections numbered]
\setbeamertemplate{subsubsection in toc}[subsubsections numbered]
\begin{frame}[allowframebreaks]{Table des matières du chapitre}
    \tableofcontents[subsectionstyle=show/show/hide,subsubsectionstyle=show/show/hide,]
\end{frame}

\section{Introduction}
\tocss
\subsection{Le contexte}
\begin{frame}{\secname : \subsecname}
    \begin{itemize}
        \item Canvas fut inventé par Apple pour Safari et Dashboard\footnote{On doit son existence à son absorption progressive. Le WhatWG a choisi de l’intégrer à HTML 5.};
        \item Figures de proue de HTML 5 \lstinline[language=html]!<canvas>!;
        \item API de dessin 2D, 3D;
        \item Alternative sérieuse et puissante à Flash ou Java;
    \end{itemize}
\end{frame}

\subsection{Possibilités}
\begin{frame}{\secname : \subsecname}
    Parmi les possibilités offertes figurent :
    \begin{itemize}
        \item des méthodes de tracé de formes géométriques (cercles, rectangles...) ;
        \item des méthodes de tracé de polygones ;
        \item des méthodes de choix de styles de couleurs et de remplissages ;
        \item des méthodes de tracé de texte ;
        \item des méthodes d’import et de manipulation d’images ;
        \item des méthodes de transformation (échelle, rotation, déplacement) qui affectent
              toute la matrice.
    \end{itemize}
\end{frame}

\subsection{Usages}
\begin{frame}{\secname : \subsecname}
    Les usages de canvas sont multiples :
    \begin{itemize}
        \item Affichage de graphiques générés dynamiquement à partir de données variables;
        \item Création de jeux;
        \item Édition d’images et de photos en ligne;
        \item Effets cosmétiques.
    \end{itemize}
\end{frame}
\section{L’élément <canvas>}
\tocss
\subsection{Zone de dessin}
\begin{frame}{\secname : \subsecname}
    \lstinputlisting[language=html, title=Une zone de dessin canvas]{./exemples/canvas1.html}
\end{frame}
\begin{frame}{\secname : \subsecname}
    \begin{itemize}
        \item \lstinline[language=sql]!<canvas>! est un élément similaire à \lstinline[language=sql]!<img>! et ne nécessite pas à tout prix de balise de fermeture.
        \item Dans la famille Mozilla, le choix de pouvoir fournir du contenu alternatif (facultatif, entre les deux balises) impose un fonctionnement avec une balise de fermeture.
        \item Cette solution fonctionne également sous WebKit.\footnote{WebKit ignorera simplement cette balise.}
    \end{itemize}
\end{frame}

\subsection{L’aide de JavaScript}
\begin{frame}{\secname : \subsecname}
    Quelques appels de fonction suffisent pour dessiner
    \lstinputlisting[language=JavaScript, title=Un simple rectangle rouge.]{./exemples/canvas1.js}
\end{frame}

\section{Coordonnées}
\tocss
\subsection{Un système}
\begin{frame}{\secname : \subsecname}
    \begin{itemize}
        \item Les coordonnées sont définies dans un système cartésien;
        \item Débutent dans le coin supérieur gauche de la zone à $(0,0)$;
        \item Toutes les valeurs sont exprimées en pixels;
        \item Deux axes : horizontal $(x)$ et vertical $(y)$.
    \end{itemize}
\end{frame}

\begin{frame}{\secname : \subsecname}
    \begin{center}
        \includegraphics{./img/Coordonnées.jpg}
    \end{center}
\end{frame}

\begin{frame}{\secname : \subsecname}
    Sauf que...
    \begin{center}
        \includegraphics{./img/Coordonnées2.jpg}
    \end{center}
\end{frame}

\begin{frame}{\secname : \subsecname}
    \begin{quote}
        Il ne reste plus qu’à découvrir l’ensemble des méthodes de dessin décrit dans la spécification.
    \end{quote}
\end{frame}


\end{document}
